% Contributions are much appreciated, in order to contribute to this project, head over to this repository:
% https://github.com/bshramin/uofa-eng-assignment

\documentclass[11pt,letterpaper]{article}
\textwidth 6.5in
\textheight 9.in
\oddsidemargin 0in
\headheight 0in
\usepackage{graphicx}
\usepackage{fancybox}
\usepackage[utf8]{inputenc}
\usepackage{epsfig,graphicx}
\usepackage{multicol,pst-plot}
\usepackage{pstricks}
\usepackage{amsmath}
\usepackage{enumitem}
\usepackage{amsfonts}
\usepackage{amssymb}
\usepackage{eucal}
\usepackage{hyperref}
\usepackage[left=2cm,right=2cm,top=2cm,bottom=2cm]{geometry}
\pagestyle{empty}
\DeclareMathOperator{\tr}{Tr}
\newcommand*{\op}[1]{\check{\mathbf#1}}
\newcommand{\bra}[1]{\langle #1 |}
\newcommand{\ket}[1]{| #1 \rangle}
\newcommand{\braket}[2]{\langle #1 | #2 \rangle}
\newcommand{\mean}[1]{\langle #1 \rangle}
\newcommand{\opvec}[1]{\check{\vec #1}}
\renewcommand{\sp}[1]{$${\begin{split}#1\end{split}}$$}

\usepackage{lipsum}

\usepackage{listings}
\usepackage{soul}
\usepackage{color}

\definecolor{codegreen}{rgb}{0,0.6,0}
\definecolor{codegray}{rgb}{0.5,0.5,0.5}
\definecolor{codepurple}{rgb}{0.58,0,0.82}
\definecolor{backcolour}{rgb}{0.95,0.95,0.92}

\lstdefinestyle{mystyle}{
	backgroundcolor=\color{backcolour},   
	commentstyle=\color{codegreen},
	keywordstyle=\color{magenta},
	numberstyle=\tiny\color{codegray},
	stringstyle=\color{codepurple},
	basicstyle=\footnotesize,
	breakatwhitespace=false,         
	breaklines=true,                 
	captionpos=b,                    
	keepspaces=true,                 
	numbers=left,                    
	numbersep=5pt,                  
	showspaces=false,                
	showstringspaces=false,
	showtabs=false,                  
	tabsize=2
}

\lstset{style=mystyle}

\begin{document}
\pagestyle{plain}


% \begin{flushright}\vspace{-5mm}
% \includegraphics[height=2cm]{logo.png}
% \end{flushright}
 
\begin{center}
\textbf{\Large DSC 210 Numerical Linear Algebra, Fall 2025} \\ \bigskip
\large{Homework Problems for Topic 2: \textit{Systems of Linear Equations}} \\  \bigskip
\begin{flushleft}
    \large{Student Name (PID):} James Doan (A15903661)
\end{flushleft}
\end{center}
\vspace{-4mm}
\rule{\linewidth}{0.1mm}
%%%%%%%%%%%%%%%%%%%%%%%%%%%%%%%%%%%%%%%%%%%%%%%%%%%%%%%%%%%%%%%%%%%%%%%%

% \bigskip
\bigskip

\begin{enumerate}

\item[] \fbox{%
\begin{minipage}{0.95\textwidth}
Write your solutions to the following problems by typing them in \LaTeX. Unless otherwise noted by the
problem's instructions, show your work and provide justification for
your answer. Homework is due via Gradescope at \textbf{6th November 2025, 11:59 PM}.
\\
\textbf{Late Policy}: If you submit your homework after the deadline we will apply a late penalty of $10\%$ per day.

\item[] \textbf{Guidelines for Homework Related Questions:}
\begin{enumerate}
    \item As a general rule, we can help you understand the homework problems and explain the material from the corresponding lectures, but we cannot give you the entire solution.
    \item Regarding debugging programming questions: We ask you to do some debugging on your own first, including printing out intermediate values in your algorithms, trying a simpler version of the problem, etc.
    \item We will not be pre-grading the homework, i.e. we won’t confirm if the answer you have is correct.
\end{enumerate}

\item[] \textbf{AI Usage Policy:}
\begin{enumerate}
    \item Code: You may use LLMs to debug your code; however, you may not use LLMs to generate your entire code, and code must be reviewed and tested.
    \item Writing: You may use LLMs to correct grammar, style and latex issues; however, you may not use LLMs to generate entire solutions, sentences or paragraphs. All writing must be in your own voice.
\end{enumerate}

\item [] \textbf{Academic Integrity Policy:}
\begin{enumerate}
    \item [] The UC San Diego Academic Integrity Policy (formerly the Policy on Integrity of Scholarship) is effective as of September 25, 2023 and applies to any cases originating on or after September 25, 2023. The university expects both faculty and students to honor the policy. For students, this means that all academic work will be done by the individual to whom it's assigned, without unauthorized aid of any kind. If violations of academic integrity occur, the same Sanctioning Guidelines apply regardless of which policy was effective for that case.
    
    For more information on how the policy is implemented, refer to the most current procedures. Remember: When in doubt about what constitutes appropriate collaboration or resource use, please ask TAs before proceeding. It's always better to clarify expectations than to risk an academic integrity violation. Academic integrity violations can have serious consequences for your academic record, and you will get zero grades.
\end{enumerate}



You can access the Homework Template using the following link: \url{https://www.overleaf.com/read/vfhcmsppvskp}
\end{minipage}}

%%%%%%%%%%%%%%%
\clearpage
\begin{enumerate}
%%%%%%%%%%%%%%%
\item[] \textbf{Question 1: Gauss elimination (20 points)} 
%%%%%%%%%%%%%%%

Use Gauss elimination to solve the following equations for $\mathbf{x}=\begin{bmatrix}x_1 & x_2 & x_3\end{bmatrix}^\top$:
\begin{align*}
2x_1 + 6x_2 +7x_3 &= -11 \\
-8x_1  +10x_2 + 3x_3 &= - 15 \\
9x_1 + 10x_2 + x_3 &= 25
\end{align*}

\item[] \textbf{Solution:}

Starting with Gaussian elimination:

\begin{align*}
	\begin{bmatrix}
	2 & 6 & 7 \\
	-8 & 10 & 3 \\
	9 & 10 & 1
	\end{bmatrix}
	\begin{bmatrix}
	x_1 \\ x_2 \\ x_3
	\end{bmatrix}
	&=
	\begin{bmatrix}
	-11 \\ -15 \\ 25
	\end{bmatrix}
	\\
	\begin{bmatrix}
	x_1 \\ x_2 \\ x_3
	\end{bmatrix}
	&=
	\begin{bmatrix}
	2 & 6 & 7 & -11 \\
	-8 & 10 & 3 & -15 \\
	9 & 10 & 1 & 25
	\end{bmatrix}
	\\ 
	\text{R1 = R1/2} &\Rightarrow
	\begin{bmatrix}
	1 & 3 & 7/2 & -11/2 \\
	-8 & 10 & 3 & -15 \\
	9 & 10 & 1 & 25
	\end{bmatrix}
	\\
	\text{R2 = R2 + 8R1} &\Rightarrow
	\begin{bmatrix}
	1 & 3 & 7/2 & -11/2 \\
	0 & 34 & 31 & -59 \\
	9 & 10 & 1 & 25
	\end{bmatrix}
	\\
	\text{R3 = R3 - 9R1} &\Rightarrow
	\begin{bmatrix}
	1 & 3 & 7/2 & -11/2 \\
	0 & 34 & 31 & -59 \\
	0 & -17 & -61/2 & 149/2
	\end{bmatrix}
	\\
	\text{R3 = R3 + R2/2} &\Rightarrow
	\begin{bmatrix}
	1 & 3 & 7/2 & -11/2 \\
	0 & 34 & 31 & -59 \\
	0 & 0 & -15 & 45
	\end{bmatrix}
\end{align*}

Continuing with backward substitution:

If \(-15x_3 = 5\) then \( x_3 = -3 \)

\( \therefore 0x_1 + 34x_2 + 31(-3) = -59 \Rightarrow x_2 = 1 \)

\( \therefore 1x_1 + 3(1) + \frac{7}{2} = -\frac{11}{2} \Rightarrow x_1 = 2 \)

\( \therefore \mathbf{x}=\begin{bmatrix}x_1 & x_2 & x_3\end{bmatrix}^\top=\begin{bmatrix}2 \\ 1 \\ -3\end{bmatrix} \hfill \square\)

\newpage

%%%%%%%%%%%%%%%
\item[] \textbf{Question 2: LU decomposition (20 points)} 
%%%%%%%%%%%%%%%

Perform LU decomposition for the matrix corresponding to the equations in \textbf{Question 1}. Using the matrices $L$ and $U$, do forward and backward substitution to solve for $\mathbf{x}$. Match your solution with that of \textbf{Question 1}.

\item[] \textbf{Solution:}

Solving for \( \mathbf{L} \) by applying the inverse of the row operations done to obtain \( \mathbf{U} \) previously (aka the upper right triangular matrix):

\begin{align*}
	L = I_3 &=
	\begin{bmatrix}
	1 & 0 & 0 \\
	0 & 1 & 0 \\
	0 & 0 & 1
	\end{bmatrix}
	\\
	\text{R1 = R1/2} &\Rightarrow
	\begin{bmatrix}
	2 & 0 & 0 \\
	0 & 1 & 0 \\
	0 & 0 & 1
	\end{bmatrix}
	\\
	\text{R2 = R2 + 8R1} &\Rightarrow
	\begin{bmatrix}
	2 & 0 & 0 \\
	-8 & 1 & 0 \\
	0 & 0 & 1
	\end{bmatrix}
	\\
	\text{R3 = R3 - 9R1} &\Rightarrow
	\begin{bmatrix}
	2 & 0 & 0 \\
	-8 & 1 & 0 \\
	9 & 0 & 1
	\end{bmatrix}
	\\
	\text{R3 = R3 + R2/2} &\Rightarrow
	\begin{bmatrix}
	2 & 0 & 0 \\
	-8 & 1 & 0 \\
	9 & -1/2 & 1
	\end{bmatrix}
\end{align*}

Solve \( \mathbf{Ly = b} \) with forward substitution:
\begin{align*}
	Ly &= b
	\\
	\begin{bmatrix}
	2 & 0 & 0 \\
	-8 & 1 & 0 \\
	9 & -1/2 & 1
	\end{bmatrix}
	\begin{bmatrix}y_1 \\ y_2 \\ y_3\end{bmatrix}
	&=
	\begin{bmatrix}-11 \\ -15 \\ 25\end{bmatrix}
\end{align*}

If \(2y_1 = -11\) then \( y_1 = -\frac{11}{2} \)

\( \therefore -8(-\frac{11}{2}) + y_2 + 0 y_3 = 15 \Rightarrow y_2 = -59 \)

\( \therefore 9(-\frac{11}{2}) + -\frac{1}{2}(-59) + y_3 = 25 \Rightarrow y_3 = 5\)

\( \therefore \mathbf{y}=\begin{bmatrix}y_1 & y_2 & y_3\end{bmatrix}^\top=\begin{bmatrix}-\frac{11}{2} \\ -59 \\ 5\end{bmatrix} \)

Solve \( \mathbf{Ux = y} \) with backward substitution:
\begin{align*}
	Ux &= y
	\\
	\begin{bmatrix}
	1 & 3 & 7/2\\
	0 & 34 & 31\\
	0 & 0 & 1
	\end{bmatrix}
	\begin{bmatrix}x_1 \\ x_2 \\ x_3\end{bmatrix}
	&=
	\begin{bmatrix}-\frac{11}{2} \\ -59 \\ 5\end{bmatrix}
\end{align*}

If \(-15x_3 = 5\) then \( x_3 = -3 \)

\( \therefore 0x_1 + 34x_2 + 31(-3) = -59 \Rightarrow x_2 = 1 \)

\( \therefore 1x_1 + 3(1) + \frac{7}{2} = -\frac{11}{2} \Rightarrow x_1 = 2 \)

\( \therefore \mathbf{x}=\begin{bmatrix}x_1 & x_2 & x_3\end{bmatrix}^\top=\begin{bmatrix}2 \\ 1 \\ -3\end{bmatrix} \hfill \square\)

\newpage

%%%%%%%%%%%%%%%
\item[] \textbf{Question 3: QR decomposition (20 points)} 
%%%%%%%%%%%%%%%

Perform QR decomposition for the matrix corresponding to the equations in \textbf{Question 1} using the Gram-Schmidt algorithm. Using the decomposition, solve for $\mathbf{x}$. Match your solution with that of \textbf{Question 1} and \textbf{Question 2}.

\textbf{Solution:}

\begin{align*}
	\mathbf{A} = \begin{bmatrix}
	2 & 6 & 7 \\
	-8 & 10 & 3 \\
	9 & 10 & 1
	\end{bmatrix}
\end{align*}

Let the three columns of this matrix be denoted as \( a_1, a_2, a_3 \).

Solving for \( q_k \):
% flalign (flush left align) requires `&\\` not `\\`
% -----------------------------------------------
\begin{flalign*}
	u_1 &= a_1 &\\
	&= (2, -8, 9)
\end{flalign*}

\begin{flalign*}
	q_1 &= \frac{u_1}{\lVert u_1 \rVert} &\\
	&= \frac{(2, -8, 9)}{\sqrt{2^2 + (-8)^2 + 9^2}} &\\
	&= \frac{(2, -8, 9)}{\sqrt{149}} &\\
	&= (\frac{2}{\sqrt{149}}, \frac{-8}{\sqrt{149}}, \frac{9}{\sqrt{149}})
\end{flalign*}

\begin{flalign*}
	u_2 &= a_2 - proj_{q_1}(a_2) \cdot a_2 &\\
	&= a_2 - (a_2 \cdot q_1)(q_1) &\\
	&= (6, 10, 10) - (6 \cdot \frac{2}{\sqrt{149}} + 10 \cdot \frac{(-8)}{\sqrt{149}} + 10 \cdot \frac{9}{\sqrt{149}})(\frac{(2, -8, 9)}{\sqrt{149}}) &\\
	&= (6, 10, 10) - (\frac{22}{149})(2, -8, 9) &\\
	&= (\frac{850}{149}, \frac{1666}{149}, \frac{1292}{149})
\end{flalign*}

\begin{flalign*}
	q_2 &= \frac{u_2}{\lVert u_2 \rVert} &\\
	&= \frac{(\frac{850}{149}, \frac{1666}{149}, \frac{1292}{149})}{\sqrt{(\frac{850}{149})^2 + (\frac{1666}{149})^2 + (\frac{1292}{149})^2}} &\\
	&= \frac{34(25, 49, 38)}{\sqrt{5,167,320}} &\\
	&= \frac{(25, 49, 38)}{\frac{\sqrt{5,167,320}}{34}} &\\
	&= \frac{(25, 49, 38)}{\sqrt{4470}} &\\
	&= (\frac{25}{\sqrt{4470}}, \frac{49}{\sqrt{4470}}, \frac{38}{\sqrt{4470}})
\end{flalign*}

\begin{flalign*}
	u_3 &= a_3 - proj_{q_1}(a_3) \cdot a_1 - proj_{q_2}(a_3) \cdot a_2 &\\
	&= a_3 - (a_3 \cdot q_1)(q_1) - (a_3 \cdot q_2)(q_2) &\\
	&= (7, 3, 1) - (7 \cdot \frac{2}{\sqrt{149}} + 3 \cdot \frac{(-8)}{\sqrt{149}} + 1 \cdot \frac{9}{\sqrt{149}})(\frac{(2, -8, 9)}{\sqrt{149}}) - (7 \cdot \frac{25}{\sqrt{4470}} + 3 \cdot \frac{49}{\sqrt{4470}} + 1 \cdot \frac{38}{\sqrt{4470}})(\frac{25, 49, 38}{4470}) &\\
	&= (7, 3, 1) - (-\frac{1}{149})(2, -8, 9) - (\frac{12}{149})(25, 49, 38) &\\
	&= (5, -1, -2)
\end{flalign*}

\begin{flalign*}
	q_3 &= \frac{u_3}{\lVert u_3 \rVert} &\\
	&= \frac{(5, -1, -2)}{\sqrt{5^2 + (-1)^2 + (-2)^2}} &\\
	&= \frac{(5, -1, -2)}{\sqrt{30}} &\\
	&= (\frac{5}{\sqrt{30}}, \frac{-1}{\sqrt{30}}, \frac{-2}{\sqrt{30}})
\end{flalign*}
% -----------------------------------------------

Q and R matricies:
% -----------------------------------------------
\begin{flalign*}
	\mathbf{Q} &= \begin{bmatrix}
	q_1 & q_2 & q_3
	\end{bmatrix} &\\
	&= \begin{bmatrix}
	\frac{2}{\sqrt{149}} & \frac{25}{\sqrt{4470}} & \frac{5}{\sqrt{30}} &\\
	\frac{-8}{\sqrt{149}} & \frac{49}{\sqrt{4470}} & \frac{-1}{\sqrt{30}} &\\
	\frac{9}{\sqrt{149}} & \frac{38}{\sqrt{4470}} & \frac{-2}{\sqrt{30}}
	\end{bmatrix}
\end{flalign*}

\begin{flalign*}
	\mathbf{R} &= \begin{bmatrix}
	a_1 \cdot q_1 & a_2 \cdot q_1 & a_3 \cdot q_1 \\
	0 & a_2 \cdot q_2 & a_3 \cdot q_2 \\
	0 & 0 & a_3 \cdot q_3 \\
	\end{bmatrix} &\\
	&= \begin{bmatrix}
	2(\frac{2}{\sqrt{149}}) + (-8)(\frac{-8}{\sqrt{149}}) + 9(\frac{9}{\sqrt{149}}) & 6(\frac{2}{\sqrt{149}}) + 10(\frac{-8}{\sqrt{149}}) + 10(\frac{9}{\sqrt{149}}) & 7(\frac{2}{\sqrt{149}}) + 3(\frac{-8}{\sqrt{149}}) + 1(\frac{9}{\sqrt{149}}) \\
	0 & 6(\frac{25}{\sqrt{4470}}) + 10(\frac{49}{\sqrt{4470}}) + 10(\frac{38}{\sqrt{4470}}) & 7(\frac{25}{\sqrt{4470}}) + 3(\frac{49}{\sqrt{4470}}) + 1(\frac{38}{\sqrt{4470}}) \\
	0 & 0 & 7(\frac{5}{\sqrt{30}}) + 3(\frac{-1}{\sqrt{30}}) + 1(\frac{-2}{\sqrt{30}})
	\end{bmatrix} &\\
	&= \begin{bmatrix}
	\sqrt{149} & \frac{22}{\sqrt{149}} & -\frac{1}{\sqrt{149}} \\
	0 & \frac{1020}{\sqrt{4470}} & \frac{360}{\sqrt{4470}} \\
	0 & 0 & \sqrt{30}
	\end{bmatrix}
\end{flalign*}
% -----------------------------------------------

To solve for x, first let \( A = QR \).
\begin{align*}
	\therefore QRx &= b \\
	\therefore Rx &= Q^\top b \\
	\therefore y &= Q^\top b
\end{align*}

Solving for the second equation above gives:

\begin{align*}
	Rx &= Q^\top b \\
	\begin{bmatrix}
	\sqrt{149} & \frac{22}{\sqrt{149}} & -\frac{1}{\sqrt{149}} \\
	0 & \frac{1020}{\sqrt{4470}} & \frac{360}{\sqrt{4470}} \\
	0 & 0 & \sqrt{30}
	\end{bmatrix}
	\begin{bmatrix}
	x_1 \\
	x_2 \\
	x_3
	\end{bmatrix}
	&= \begin{bmatrix}
	\frac{2}{\sqrt{149}} & \frac{-8}{\sqrt{149}} & \frac{9}{\sqrt{149}} \\
	\frac{25}{\sqrt{4470}} & \frac{49}{\sqrt{4470}} & \frac{38}{\sqrt{4470}} \\
	\frac{5}{\sqrt{30}} & \frac{-1}{\sqrt{30}} & \frac{-2}{\sqrt{30}}
	\end{bmatrix}
	\begin{bmatrix} -11 \\ -15 \\ 25 \end{bmatrix} \\
	\begin{bmatrix}
	\sqrt{149} & \frac{22}{\sqrt{149}} & -\frac{1}{\sqrt{149}} \\
	0 & \frac{1020}{\sqrt{4470}} & \frac{360}{\sqrt{4470}} \\
	0 & 0 & \sqrt{30}
	\end{bmatrix}
	\begin{bmatrix}
	x_1 \\
	x_2 \\
	x_3
	\end{bmatrix}
	&= \begin{bmatrix}
	\frac{323}{\sqrt{149}} \\ 
	\frac{-60}{\sqrt{4470}} \\ 
	\frac{-90}{\sqrt{30}}
	\end{bmatrix}
\end{align*}

Continuing with backward substitution:

If \( \sqrt{30}x_3 = \frac{-90}{\sqrt{30}} \) then \(x_3 = -3 \)

\( \therefore 0x_1\frac{1020}{\sqrt{4470}}x_2 + \frac{360}{\sqrt{4470}}(-3) = \frac{-60}{\sqrt{4470}} \Rightarrow x_2 = 1 \)

\( \therefore \sqrt{149}x_1 + \frac{22}{\sqrt(149)}(1) - \frac{1}{\sqrt{149}}(-3) = \frac{323}{\sqrt{149}} \Rightarrow x_1 = 2 \)

\( \therefore \mathbf{x}=\begin{bmatrix}x_1 & x_2 & x_3\end{bmatrix}^\top=\begin{bmatrix}2 \\ 1 \\ -3\end{bmatrix} \hfill \square \)

\newpage

%%%%%%%%%%%%%%%
\item[] \textbf{Question 4: Gram-Schmidt process (20 points)} 
%%%%%%%%%%%%%%%

Show that the residual vector $\mathbf{a}_i^{\perp}$ is orthogonal to $\mathbf{q}_1, \mathbf{q}_2,\dots, \mathbf{q}_{i-1}$ in the Gram-Schmidt process.

\textbf{Solution:}

By definition of a projection \( \text{proj}_{q_k}(\mathbf{a}_i) \):

\begin{align*}
	\text{proj}_{q_k}(\mathbf{a}_i) = \frac{\mathbf{q}_k \cdot \mathbf{a}_i}{\mathbf{q}_k \cdot \mathbf{q}_k} \mathbf{q}_k
\end{align*}

Let the residual vector \( \mathbf{a}_i^{\perp} \) be:

\begin{align*}
	\mathbf{a}_i^{\perp} = \mathbf{a}_i - \sum_{k=1}^{i-1} \text{proj}_{q_k}(\mathbf{a}_i)
\end{align*}

By definition of orthogonality, \( a \cdot b = 0 \),  To prove this, we can replace \( a \) and \( b \) with the definitions of \( \mathbf{j} = \mathbf{q}_1, \mathbf{q}_2,\dots, \mathbf{q}_{i-1} \) and the residual vector, respectively:

\begin{align*}
	\mathbf{q}_k \cdot \mathbf{a}_i^{\perp}
\end{align*}

\begin{align*}
	\mathbf{q}_k \cdot \mathbf{a}_i - \sum_{k=1}^{i-1} \text{proj}_{q_k}(\mathbf{a}_i)
\end{align*}

By substituting in the definition of the projection into this equation, it is shown that

\begin{align*}
	\mathbf{q}_k \cdot \mathbf{a}_i - \sum_{j=1}^{i-1} \frac{\mathbf{q}_j \cdot \mathbf{a}_i}{\mathbf{q}_j \cdot \mathbf{q}_j} \mathbf{q}_j
\end{align*}

Rearranging gives:

\begin{align*}
	\mathbf{a}_i - \sum_{j=1}^{i-1} \frac{\mathbf{q}_j \cdot \mathbf{a}_i}{\mathbf{q}_j \cdot \mathbf{q}_j} \mathbf{q}_j \cdot \mathbf{q}_k
\end{align*}

Now, since orthogonality means the dot product of \( \mathbf{q}_j \) and \( \mathbf{q}_k \) must be equal to 0 \( \forall \mathbf{j} \). This implies that:

\begin{align*}
	\mathbf{q}_j = \mathbf{q}_k \forall \text{j, k}
\end{align*}

Therefore,

\begin{align*}
	\mathbf{q}_k \cdot \mathbf{a}_i^{\perp} = \mathbf{a}_i - \sum_{k=1}^{i-1} \frac{\mathbf{q}_k \cdot \mathbf{a}_i}{\mathbf{q}_k \cdot \mathbf{q}_k} \mathbf{q}_k \cdot \mathbf{q}_k
\end{align*}

This is true for all values of j, now denoted as j such that \( j = k = 1, 2, \dots, i-1 \), therefore the residual vector \( \mathbf{a}_i^{\perp} \) is orthogonal to \( \mathbf{q}_1, \mathbf{q}_2,\dots, \mathbf{q}_{i-1} \hfill \square\).

\newpage

\item[] \textbf{Question 5: Rank deficient matrices (20 points)}

Compute the QR decomposition of following two matrices using Gram-Schmidt process and properties of the matrix $\mathbf{Q}$.
\begin{enumerate}
    \item $\mathbf{A} = \begin{bmatrix}
-1 & 1 & 1\\
-1 & -1 & 1\\
1 & 1 & -1
\end{bmatrix}$
\item $\mathbf{B} = \begin{bmatrix}
-1 & -1 & -1\\
1 & 1 & 1\\
-1 & -1 & -1
\end{bmatrix}$
\end{enumerate}
Note: matrices $\mathbf{Q}$ and $\mathbf{R}$ must be square matrices.

\textbf{Solution:}

% -----------------------------------------------
\begin{align*}
	\mathbf{A} = \begin{bmatrix}
	-1 & 1 & 1\\
	-1 & -1 & 1\\
	1 & 1 & -1
	\end{bmatrix}
\end{align*}

Let the three columns of this matrix be denoted as \( a_1, a_2, a_3 \). Note: Rank(A) = 2; column 3 linearly dependent on column 1.

\begin{flalign*}
	u_1 &= a_1 &\\
	&= (-1, -1, 1)
\end{flalign*}

\begin{flalign*}
	q_1 &= \frac{u_1}{\lVert u_1 \rVert} &\\
	&= \frac{(-1, -1, 1)}{\sqrt{(-1)^2 + (-1)^2 + (1)^2}} &\\
	&= \frac{(-1, -1, 1)}{\sqrt{3}} &\\
	&= (\frac{-1}{\sqrt{3}}, \frac{-1}{\sqrt{3}}, \frac{1}{\sqrt{3}})
\end{flalign*}

\begin{flalign*}
	u_2 &= a_2 - proj_{q_2}(a_1) \cdot a_2 &\\
	&= a_2 - (a_2 \cdot q_1)(q_1) &\\
	&= (1, -1, 1) - (1 \cdot \frac{-1}{\sqrt{3}} + -1 \cdot \frac{(-1)}{\sqrt{1439}} + 1 \cdot \frac{1}{\sqrt{3}})(\frac{(1, -1, 1)}{\sqrt{149}}) &\\
	&= (1, -1, 1) - (\frac{1}{3})(-1, -1, 1) &\\
	&= (\frac{4}{3}, \frac{-2}{3}, \frac{2}{3})
\end{flalign*}

\begin{flalign*}
	q_2 &= \frac{u_2}{\lVert u_2 \rVert} &\\
	&= \frac{(\frac{4}{3}, \frac{-2}{3}, \frac{2}{3})}{\sqrt{(\frac{4}{3})^2 + (\frac{-2}{3})^2 + (\frac{2}{3})^2}} &\\
	&= \frac{(\frac{4}{3}, \frac{-2}{3}, \frac{2}{3})}{\sqrt{(4)^2 + (-2)^2 + (2)^2}} &\\
	&= (\frac{2}{\sqrt{6}}, \frac{-1}{\sqrt{6}}, \frac{1}{\sqrt{6}})
\end{flalign*}

Due to the rank deficiency of A, \( q_3 \) must be orthogonal to \( q_1, q_2 \) which can be done with the cross product. It must also normalized like \( q_1, q_2 \).

\begin{flalign*} % take advantage of rank deficiency
	q_3 &= \frac{q_1 \times q_2}{\lVert q_1 \times q_2 \rVert} &\\
	&= \frac{\begin{vmatrix}
	i & j & k \\
	\frac{-1}{\sqrt{3}} & \frac{-1}{\sqrt{3}} & \frac{1}{\sqrt{3}} \\
	\frac{2}{\sqrt{6}} & \frac{-1}{\sqrt{6}} & \frac{1}{\sqrt{6}}
	\end{vmatrix}}{\lVert q_1 \times q_2 \rVert} &\\
	&= \frac{i(\frac{-1}{\sqrt{3}} \cdot \frac{1}{\sqrt{6}} - \frac{1}{\sqrt{3}} \cdot \frac{1}{\sqrt{6}}) - j(\frac{-1}{\sqrt{3}} \cdot \frac{1}{\sqrt{6}} - \frac{1}{\sqrt{3}} \cdot \frac{2}{\sqrt{6}}) + k(\frac{-1}{\sqrt{3}} \cdot \frac{-1}{\sqrt{6}} - \frac{-1}{\sqrt{3}} \cdot \frac{2}{\sqrt{6}})}{\lVert q_1 \times q_2 \rVert} &\\
	&= \frac{(0, \frac{1}{\sqrt{2}}, \frac{1}{\sqrt{2}})}{{\lVert q_1 \times q_2 \rVert}} &\\
	&= \frac{(0, \frac{1}{\sqrt{2}}, \frac{1}{\sqrt{2}})}{{\sqrt{0^2 + (\frac{1}{\sqrt{2}})^2 + (\frac{1}{\sqrt{2}})^2}}} &\\
	&= (0, \frac{1}{\sqrt{2}}, \frac{1}{\sqrt{2}})
\end{flalign*}

Therefore,

\begin{flalign*}
	\mathbf{Q} &= \begin{bmatrix}
		q_1 & q_2 & q_3
	\end{bmatrix} \\
	& = \begin{bmatrix}
	\frac{-1}{\sqrt{3}} & \frac{2}{\sqrt{6}} & 0 \\
	\frac{-1}{\sqrt{3}} & \frac{-1}{\sqrt{6}} & \frac{1}{\sqrt{2}} \\
	\frac{1}{\sqrt{3}} & \frac{1}{\sqrt{6}} & \frac{1}{\sqrt{2}}
	\end{bmatrix}
\end{flalign*}

\begin{flalign*}
	% rank(A) = 2, so 2 rows
	\mathbf{R} &= \begin{bmatrix}
	a_1 \cdot q_1 & a_2 \cdot q_1 & a_3 \cdot q_1 \\
	0 & a_2 \cdot q_2 & a_3 \cdot q_2 \\
	0 & 0 & a_3 \cdot q_3 \\
	\end{bmatrix} \\
	&= \begin{bmatrix}
	(-1)(\frac{-1}{\sqrt{3}}) + (-1)(\frac{-1}{\sqrt{3}}) + 1(\frac{1}{\sqrt{3}}) & 1(\frac{-1}{\sqrt{3}}) + (-1)(\frac{-1}{\sqrt{3}}) + 1(\frac{1}{\sqrt{3}}) & 1(\frac{-1}{\sqrt{3}}) + 1(\frac{-1}{\sqrt{3}}) + (-1)(\frac{1}{\sqrt{3}}) \\
	0 & 1(\frac{2}{\sqrt{6}}) + (-1)(\frac{-1}{\sqrt{6}}) + 1(\frac{1}{\sqrt{6}}) & 1(\frac{2}{\sqrt{6}}) + 1(\frac{-1}{\sqrt{6}}) + (-1)(\frac{1}{\sqrt{6}}) \\
	0 & 0 & 1(0) + 1(\frac{1}{\sqrt{2}}) + (-1)(\frac{1}{\sqrt{2}})
	\end{bmatrix} \\
	&= \begin{bmatrix}
	\sqrt{3} & \frac{1}{\sqrt{3}} & \frac{-3}{\sqrt{3}} \\
	0 & \frac{4}{\sqrt{6}} & 0 \\
	0 & 0 & 0
	\end{bmatrix} \\
\end{flalign*}

% -----------------------------------------------
\begin{align*}
	\mathbf{B} = \begin{bmatrix}
	-1 & -1 & -1\\
	1 & 1 & 1\\
	-1 & -1 & -1
	\end{bmatrix}
\end{align*}

Let the three columns of this matrix be denoted as \( a_1, a_2, a_3 \). Note: Rank(B) = 1; column 2 and column 3 linearly dependent on column 1.

\begin{flalign*}
	u_1 &= a_1 &\\
	&= (-1, 1, -1)
\end{flalign*}

\begin{flalign*}
	q_1 &= \frac{u_1}{\lVert u_1 \rVert} &\\
	&= \frac{(-1, 1, -1)}{\sqrt{(-1)^2 + (1)^2 + (-1)^2}} &\\
	&= \frac{(-1, 1, -1)}{\sqrt{3}} &\\
	&= (\frac{-1}{\sqrt{3}}, \frac{1}{\sqrt{3}}, \frac{-1}{\sqrt{3}})
\end{flalign*}

Since Rank(B) = 1, \( q_2 \) must be orthogonal to \( q_1 \).  Therefore, let \( a_2 \) be arbitrarily defined:

\begin{flalign*}
	a_2 = (1, 0, 0)
\end{flalign*}

\begin{flalign*}
	u_2 &= a_2 - proj_{q_2}(a_1) \cdot a_2 &\\
	&= a_2 - (a_2 \cdot q_1)(q_1) &\\
	&= (1, 0, 0) - (1 \cdot \frac{-1}{\sqrt{3}} + 0 \cdot \frac{(1)}{\sqrt{3}} + 0 \cdot \frac{-1}{\sqrt{3}})(\frac{(-1, 1, -1)}{\sqrt{3}}) &\\
	&= (1, 0, 0) - (\frac{-1}{3})(-1, 1, -1) &\\
	&= (\frac{2}{3}, \frac{1}{3}, \frac{-1}{3})
\end{flalign*}

\begin{flalign*}
	q_2 &= \frac{u_2}{\lVert u_2 \rVert} &\\
	&= \frac{(\frac{2}{3}, \frac{1}{3}, \frac{-1}{3})}{\sqrt{(\frac{2}{3})^2 + (\frac{1}{3})^2 + (\frac{-1}{3})^2}} &\\
	&= \frac{(\frac{2}{3}, \frac{1}{3}, \frac{-1}{3})}{\sqrt{6}} &\\
	&= (\frac{2}{\sqrt{6}}, \frac{1}{\sqrt{6}}, \frac{-1}{\sqrt{6}})
\end{flalign*}

Due to the rank deficiency of B, \( q_3 \) must be orthogonal to \( q_1, q_2 \) which can be done with the cross product. It must also normalized like \( q_1, q_2 \).

\begin{flalign*} % take advantage of rank deficiency
	q_3 &= \frac{q_1 \times q_2}{\lVert q_1 \times q_2 \rVert} &\\
	&= \frac{\begin{vmatrix}
	i & j & k \\
	\frac{-1}{\sqrt{3}} & \frac{1}{\sqrt{3}} & \frac{-1}{\sqrt{3}} \\
	\frac{2}{\sqrt{6}} & \frac{1}{\sqrt{6}} & \frac{-1}{\sqrt{6}}
	\end{vmatrix}}{\lVert q_1 \times q_2 \rVert} &\\
	&= \frac{i(\frac{1}{\sqrt{3}} \cdot \frac{-1}{\sqrt{6}} - \frac{-1}{sqrt{3}} \cdot \frac{1}{\sqrt{6}}) - j(\frac{-1}{\sqrt{3}} \cdot \frac{-1}{\sqrt{6}} - \frac{-1}{\sqrt{3}} \cdot \frac{2}{\sqrt{6}}) + k(\frac{-1}{\sqrt{3}} \cdot \frac{1}{\sqrt{6}} - \frac{1}{\sqrt{3}} \cdot \frac{2}{\sqrt{6}})}{\lVert q_1 \times q_2 \rVert} &\\
	&= \frac{(0, \frac{-1}{\sqrt{2}}, \frac{-1}{\sqrt{2}})}{{\lVert q_1 \times q_2 \rVert}} &\\
	&= \frac{(0, \frac{-1}{\sqrt{2}}, \frac{-1}{\sqrt{2}})}{{\sqrt{0^2 + (\frac{-1}{\sqrt{2}})^2 + (\frac{-1}{\sqrt{2}})^2}}} &\\
	&= (0, \frac{-1}{\sqrt{2}}, \frac{-1}{\sqrt{2}})
\end{flalign*}

Therefore,

\begin{flalign*}
	\mathbf{Q} &= \begin{bmatrix}
		q_1 & q_2 & q_3
	\end{bmatrix} \\
	& = \begin{bmatrix}
	\frac{-1}{\sqrt{3}} & \frac{2}{\sqrt{6}} & 0 \\
	\frac{1}{\sqrt{3}} & \frac{1}{\sqrt{6}} & \frac{-1}{\sqrt{2}} \\
	\frac{-1}{\sqrt{3}} & \frac{-1}{\sqrt{6}} & \frac{-1}{\sqrt{2}}
	\end{bmatrix}
\end{flalign*}

\begin{flalign*}
	% rank(B) = 1, so 1 row
	\mathbf{R} &= \begin{bmatrix}
	a_1 \cdot q_1 & a_2 \cdot q_1 & a_3 \cdot q_1 \\
	0 & a_2 \cdot q_2 & a_3 \cdot q_2 \\
	0 & 0 & a_3 \cdot q_3 \\
	\end{bmatrix} \\
	&= \begin{bmatrix}
	(-1)(\frac{-1}{\sqrt{3}}) + 1(\frac{1}{\sqrt{3}}) + (-1)(\frac{-1}{\sqrt{3}}) & (-1)(\frac{-1}{\sqrt{3}}) + 1(\frac{1}{\sqrt{3}}) + (-1)(\frac{-1}{\sqrt{3}}) & (-1)(\frac{-1}{\sqrt{3}}) + 1(\frac{1}{\sqrt{3}}) + (-1)(\frac{-1}{\sqrt{3}}) \\
	0 & (-1)(\frac{2}{\sqrt{6}}) + 1(\frac{1}{\sqrt{6}}) + (-1)(\frac{-1}{\sqrt{6}}) & (-1)(\frac{2}{\sqrt{6}}) + 1(\frac{1}{\sqrt{6}}) + (-1)(\frac{-1}{\sqrt{6}}) \\
	0 & 0 & (-1)(0) + 1(\frac{-1}{\sqrt{2}}) + (-1)(\frac{-1}{\sqrt{2}})
	\end{bmatrix} \\
	&= \begin{bmatrix}
	\sqrt{3} & \sqrt{3} & \sqrt{3} \\
	0 & 0 & 0 \\
	0 & 0 & 0
	\end{bmatrix} \\
\end{flalign*}

% -----------------------------------------------

\end{enumerate}
\end{enumerate}
\end{document}